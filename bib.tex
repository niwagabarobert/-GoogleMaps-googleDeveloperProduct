\documentclass[a4paper,10pt]{article}
\usepackage[utf8]{inputenc}

\usepackage[square,numbers]{natbib}
\bibliographystyle{unsrtnat}

\title{Google \texttt{Maps} }
\author{Michał Konarski and Wojciech}

\begin{document}
	
	\maketitle
	

\section*{Abstract}
Navigation system is very Essential for every person, let it be travelling from one place to another or to find the distance between multiple locations or to get information about a place.\cite{GoogleAPI,Washington}
\section*{Introduction}
In today's life, Time and Money are very important; no one wants to waste their time and money, be it travelling on a long route, instead of a short route. or it can be giving extra money for public transportation as you were unaware of the cost of the transportation. Google maps has been a great application the travellers as it provides the user with the route, they want to travel. The user can find the route from source to destination and even Google maps can guide those using GPS to reach their destination. Google maps help users to choose different types of transportation i.e. bus, train, car, etc..\cite{GoogleAPI}
\paragraph{}

Google maps is a web based navigation system developed by Google. Google maps includes broad, precise maps in 210 countries and territories. It allows the users to search for different places around the world. \cite{GoogleAPI}, It also provides some information about different place which the user wants. Google maps is used for getting locations of different city. The user can get direction for another location with respect to his own location. Google maps gives different options to the user to select their mode of transportation i.e. Bus, Train and walking. Google maps also gives the distance and time for traveling one place to another to the user..\cite{Brisbane}


\section*{Conclusions}
When You decide to use Google Maps API in Your application, then You should come up with o good plan of placing the overlays on the map and handling events. It is most of the work You will have to do. As it was shown in this paper, building web applications in.\cite{Ijarcce} NET technology based on Google Maps service, is not a difficult process. It is basically about the communication between JavaScript\cite{Springer} and. NET and it can be easily established with usage of a WebService, where data serialization is done automatically.\cite{Doe:2009:Online}


	
	\medskip
	
	\bibliography{sample}
	
\end{document}
